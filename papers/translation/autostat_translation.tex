% Use the following line _only_ if you're still using LaTeX 2.09.
%\documentstyle[icml2014,epsf,natbib]{article}
% If you rely on Latex2e packages, like most modern people use this:
\documentclass{article}

% use Times
\usepackage{times}
% For figures
\usepackage{graphicx} % more modern
%\usepackage{epsfig} % less modern
\usepackage{subfigure} 

% For citations
\usepackage{natbib}

% For algorithms
\usepackage{algorithm}
\usepackage{algorithmic}

% As of 2011, we use the hyperref package to produce hyperlinks in the
% resulting PDF.  If this breaks your system, please commend out the
% following usepackage line and replace \usepackage{icml2014} with
% \usepackage[nohyperref]{icml2014} above.
\usepackage{hyperref}

% Packages hyperref and algorithmic misbehave sometimes.  We can fix
% this with the following command.
\newcommand{\theHalgorithm}{\arabic{algorithm}}

% Employ the following version of the ``usepackage'' statement for
% submitting the draft version of the paper for review.  This will set
% the note in the first column to ``Under review.  Do not distribute.''
\usepackage{format/icml2014} 
% Employ this version of the ``usepackage'' statement after the paper has
% been accepted, when creating the final version.  This will set the
% note in the first column to ``Proceedings of the...''
%\usepackage[accepted]{icml2014}

\usepackage{times}
\usepackage{hyperref}
\usepackage{url}
\usepackage{color}
\usepackage{preamble}
\definecolor{mydarkblue}{rgb}{0,0.08,0.45}
\hypersetup{ %
    pdftitle={},
    pdfauthor={},
    pdfsubject={},
    pdfkeywords={},
    pdfborder=0 0 0,
    pdfpagemode=UseNone,
    colorlinks=true,
    linkcolor=mydarkblue,
    citecolor=mydarkblue,
    filecolor=mydarkblue,
    urlcolor=mydarkblue,
    pdfview=FitH}
    
    
\usepackage{amsmath, amsfonts, bm, lipsum, capt-of}
\usepackage{natbib, xcolor, wrapfig, booktabs, multirow, caption}
\DeclareCaptionType{copyrightbox}
\usepackage{float}

%\renewcommand{\baselinestretch}{0.99}

\def\ie{i.e.\ }
\def\eg{e.g.\ }
\let\oldemptyset\emptyset
\let\emptyset\varnothing

%\author{
%James Robert Lloyd\\
%University of Cambridge\\
%Department of Engineering\\
%\texttt{jrl44@cam.ac.uk}
%\And
%David Duvenaud\\
%University of Cambridge \\
%Department of Engineering \\
%\texttt{dkd23@cam.ac.uk}
%\And
%Roger Grosse\\
%M.I.T.\\
%Brain and Cognitive Sciences \\
%\texttt{rgrosse@mit.edu}
%\And
%Joshua B. Tenenbaum\\
%M.I.T.\\
%Brain and Cognitive Sciences \\
%\texttt{jbt@mit.edu}
%\And
%Zoubin Ghahramani\\
%University of Cambridge \\
%Department of Engineering \\
%\texttt{zoubin@eng.cam.ac.uk}
%}

\newcommand{\fix}{\marginpar{FIX}}
\newcommand{\new}{\marginpar{NEW}}

\setlength{\marginparwidth}{0.9in}
\input{include/commenting.tex}

%% For submission, make all render blank.
%\renewcommand{\LATER}[1]{}
%\renewcommand{\fLATER}[1]{}
%\renewcommand{\TBD}[1]{}
%\renewcommand{\fTBD}[1]{}
%\renewcommand{\PROBLEM}[1]{}
%\renewcommand{\fPROBLEM}[1]{}
%\renewcommand{\NA}[1]{#1}  % Note, NA's pass through!


% The \icmltitle you define below is probably too long as a header.
% Therefore, a short form for the running title is supplied here:
\icmltitlerunning{Automatically generated natural-language descriptions of time-series}

\begin{document} 

\twocolumn[
\icmltitle{Automatically generated natural-language descriptions of time-series}

% It is OKAY to include author information, even for blind
% submissions: the style file will automatically remove it for you
% unless you've provided the [accepted] option to the icml2014
% package.
\icmlauthor{Your Name}{email@yourdomain.edu}
\icmladdress{Your Fantastic Institute,
            314159 Pi St., Palo Alto, CA 94306 USA}
\icmlauthor{Your CoAuthor's Name}{email@coauthordomain.edu}
\icmladdress{Their Fantastic Institute,
            27182 Exp St., Toronto, ON M6H 2T1 CANADA}

% You may provide any keywords that you 
% find helpful for describing your paper; these are used to populate 
% the "keywords" metadata in the PDF but will not be shown in the document
\icmlkeywords{}

\vskip 0.3in
]

\begin{abstract} 
One of the broad aims of statistics is to produce an accurate and concise description of data.
Parametric regression typically produces a very concise description of a data set but can be inaccurate when data deviates from the strong assumptions of the parametric form of the model.
In contrast, nonparametric regression methods are typically capable of providing accurate descriptions of many data sets but at the expense of a succinct description.
To address this apparent antagonism, we demonstrate that a recently introduced nonparametric model-construction method allows for the automatic description of a data set in natural-language.
We also extend the class of models that can be produced by this model-construction method to include a wide variety of regression motifs; evidenced by the automatically produced descriptions of these models.
We demonstrate this procedure on time-series, showing that the automatically constructed models can accurately describe the data and also produce detailed descriptions in simple natural language.
\end{abstract} 

\allowdisplaybreaks

\section{Introduction}

Simple parametric regression models such as linear regression (cite a reference book) are usually easily interpretable, requiring only a table of regression coefficients and a few other parameters to describe the model.
In contrast, non-parametric models typically require graphical descriptions of their fit since any text description would be overly simplistic (\eg `a smooth function') and fitted parameters correspond only to very high level concepts such as bandwidths or typical lengthscales.
%, since their assumptions may be difficult to check, and their predictive implications difficult to explain.

Recently, a method for searching over a large class of structured nonparametric regression models \citep{DuvLloGroetal13} was demonstrated to decompose time series into highly interpretable additive components.
In this initial work, the fit of the model was described post-hoc by the authors, explaining how the learned structure of the model corresponded to the patterns in the additive components.
%Recent advances in nonparametric regression have demonstrated the fully automatic construction of structured nonparametric regression models \cite{grosse2012exploiting, DuvLloGroetal13}.

%These automatically-constructed models have been used for extrapolation without further human intevention, but for model-checking or dataset exploration, extrapolations alone are insufficient.  
%Fortunately, the generated models are sufficiently structured to capture human-interpretable features in a given dataset, and the compositional nature of this structure allows for simple translation into natural language.

We extend this work by demonstrating that the structure of the models searched over allows for the automatic natural-language description of patterns within a data set.
We also expand and improve the components used in the model-construction process to improve the expressivity and interpretability of the models.
This allows for the procedure to fit and describe many well established modelling motifs including, linear regression (cite), fourier analysis (cite), changepoints (cite), heteroskedasiticty (cite), trend periodic short term (cite) and nonparametirc additve / MKL (cite).

This manuscript exhibits extracts of the models and natural-language reports which are automatically generated by our procedure.
The automatically generated text descriptions of the components of the models clearly communicate interpretable features of the data.
%This paper demonstrates a procedure for automatically generating a human-readable report, examining and explaining the properties of a given nonparametric model on a given dataset.
The supplementary material to this paper is a pair of complete reports generated by our method.
%These reports summarize and describe datasets modeled by a Gaussian process with a complicated kernel.
%The different components of the kernel correspond to different features of the dataset, and the report discusses in detail the properties and significance of each component.
%Detailed descriptions of the model construction and summarization procedure are deferred to future publications.

%Section \ref{sec:gpss} gives an overview of the structure search method which produces the models that our procedure summarizes.  Section \ref{sec:method} describes our method for generating these reports, and section \ref{sec:example} examines in detail the report included in the supplementary material.

\section{Background: Gaussian Process Structure Search (GPSS)}
\label{sec:gpss}

%\subsection{Gaussian process regression}
Gaussian processes (\gp{}) \citep{rasmussen38gaussian} can be used to perform a Bayesian nonparametric regression analysis.
A \gp{} regression model uses a kernel function, $\kernel$, to define the covariance between any two function values, $\outputVar, \outputVar'$ evaluated at two inputs, $\inputVar,\inputVar'$ \ie ${\textrm{Cov}(\outputVar, \outputVar') = \kernel(\inputVar,\inputVar')}$.
%The kernel determines which sorts of structures the model places most of its probability mass upon, and in effect determines the generalization properties of the model.
The kernel specifies which structures are likely under the \gp{} prior, which in turn determines the generalisation properties of the model.

Different kernels can express a wide variety of covariance structures, such as local smoothness or periodicity.
%, and changepoints.
New kernels can be constructed by taking the product of a set of base kernels to express richer structures, such as functions which are locally periodic, or heteroskedastic.

The Gaussian process structure search (GPSS) procedure \citep{DuvLloGroetal13} searches over sums and products of a set of simple base kernels to produce an appropriate model for a given data set.
It was demonstrated that when applied to time-series the learned kernel structures could be used to decompose the learned regression function into interpretable components.

For example, when applied to airline passenger data (cite Box, see section~\ref{sec:airline}), the syntax of the learned kernel structure was (refer to description of acronyms)
\begin{equation}
\kSE \times \big( \kLin + \kLin \times ( \kPer + \kRQ ) \big)
\end{equation}
which can be distributed into a sum of products
\begin{equation}
(\kSE \times \kLin) + (\kSE \times \kLin \times \kPer) + (\kSE \times \kLin \times \kRQ).
\end{equation}
It was noted that this sum decomposed the regression function into three additive components and the second component was described as periodic with linearly increasing amplitude.
It should not be surprising that this description could have been deduced from the $\kLin$ and $\kPer$ kernels comprising the second product above.

Building upon this observation we demonstrate how to translate arbitrary sums and products of kernels in section~\ref{sec:translation}.
Before that, we introduce some extensions and improvements to the base kernels and generative grammar.

\subsection{Notation and remarks}

Introduce standard notation here to avoid confusion later.

Mention that we restrict to 1-$d$ mostly for simplicity, but the idea has already been shown to work on higher dimensional data.

\section{An improved and expanded language of statistical models}
\label{sec:improvements}

The initial work of \cite{DuvLloGroetal13} constructed the search space of GPSS via four base kernels and a generative grammar applying addition and multiplication operators.
Together, they define a language of kernels which in turn defines a language of regression models.

In this section we extend and modify this language.
In section~\ref{sec:translation} we demonstrate how all sentences in this language can be translated into natural-language, using elementary properties of kernels to reduce this task to finite or simple sub-problems.

\subsection{Changepoints}

The original language of statistical models used by GPSS was incapable of accurately describing the covariance structure exhibited in a solar irradiance data set (see section~\ref{sec:solar}).
The data set shows a clear non-linear non-stationarity; the original language of GPSS had no way of expressing this covariance structure.

A succinct extension to the modelling language to capture this phenomenon is to include the concept of changepoints (cite Mike Osbourne + others more generic papers?).
Suppose that $f_1(x) \dist \gp{}(0, \kernel_1)$ and $f_2(x) \dist \gp{}(0, \kernel_2)$ and define
\begin{equation}
f(x) = (1-\sigma(x))f_1(x) + \sigma(x)f_2(x)
\end{equation}
where $\sigma(x)$ is a sigmoid function varying between 0 and 1 \ie the function $f$ transitions between functions $f_1$ and $f_2$.
Then $f(x) \dist \gp{}(0,\kernel)$ where
\begin{equation}
\kernel(x,x') = (1-\sigma(x))k_1(x,x')(1-\sigma(x')) + \sigma(x)k_2(x)\sigma(x')
\label{eq:cp}
\end{equation}
(see section~\ref{sec:translation} for the necessary properties to prove this identity).

We represent this new operation in the grammar as the changepoint operator $\kCP(\kernel_1, \kernel_2)$.
We also introduce a change window operator, $\kCW(.,.)$, by replacing the sigmoids with products of sigmoids (resulting in a smooth `hat' function).

\subsection{Heteroskedasticity}

We introduce the white noise kernel, $\kWN$, as a base kernel.
When multiplied by linear kernels and changepoints this can express polynomially varying standard deviation as well as switching between different noise regimes.
The original GPSS procedure always assumed homoscedastic noise.

\subsection{An improved periodic kernel}

We introduce a new periodic kernel defined by
\begin{equation}
\kernel_\kPer(x, x') = \kernel_\kPer(\tau) =  \sigma^2\frac{\exp\left(\frac{\cos\frac{2 \pi \tau}{p}}{\ell^2}\right) - I_0\left(\frac{1}{\ell^2}\right)}{\exp\left(\frac{1}{\ell^2}\right) - I_0\left(\frac{1}{\ell^2}\right)}
\label{eq:periodic}
\end{equation}
where $\tau := x - x'$ and $I_0$ is the modified Bessel function of the first kind of order 0.
It is linearly related to the original periodic kernel (cite various) but has the following elegant properties.

\paragraph{Cosine limit}

A simple application of l'H\^opital's rule shows that
\begin{equation}
\kernel_\kPer(\tau) \to \sigma^2\cos\left(\frac{2 \pi \tau}{p}\right) \quad \textrm{as} \quad\ell \to \infty.
\end{equation}
It was recently demonstrated (cite Andrew) that any stationary kernel can be arbitrarily well approximated by kernels with syntax
\begin{equation}
\sum \kSE \times \cos
\end{equation}
by appealing to Bochner's theorem (cite).
By using this new periodic kernel our language of kernels also attains this completeness property.

\paragraph{No power at zero frequency}

The kernel defined by equation~\eqref{eq:periodic} has a Fourier transform of zero when evaluated at zero.
This means that functions drawn from a \gp{} with this kernel will have zero mean.
In particular,
\begin{equation}
\kernel^\textrm{Old}_\kPer(\tau) = c + m\kernel_\kPer(\tau)
\end{equation}
where $c$ and $m$ are constants.
This shows that the previous periodic kernel defined a prior over zero mean periodic function plus an independent constant function.

In preliminary testing it was noticed that this a priori independent constant could result in near perfect anti-correlation with other components a posteriori, introducing unnecessary uncertainty about individual components.
Indeed, using this new kernel, the large posterior variance of the periodic component in the airline data seen in \cite{DuvLloGroetal13} is seen to be an artefact of the constant component in the periodic kernel when compared to (refer to decompositions in this paper).

Cite the recent work by Sheffield on construction of orthogonal periodic RKHS for comparison.

\subsection{A separate constant kernel}

Following the a posteriori anti-correlation observation above, we also remove the constant component from $\kLin$ and introduce the constant function as a separate base kernel.

\subsection{No rational quadratic}

The work of \cite{DuvLloGroetal13} also included the rational quadratic kernel which can be represented as a mixture of $\kSE$ with different lengthscales.
In preliminary investigations we found that this mixture of lengthscales could result in individual components expressing smooth trends and noise like components simultaneously.
We have therefore removed this kernel on the grounds of interpretability.

\subsection{A broad language of statistical models}

With these additions to the modelling language we can express a wide variety of regression motifs

linear regression (cite), fourier analysis (cite), changepoints (cite), heteroskedasiticty (cite), trend periodic short term (cite) and nonparametirc additve (cite) and andrews work (cite) - do we also want multidimensional things?

In section~\ref{sec:example_translation} we demonstrate the natural-language descriptions of some of these kernel forms.

\section{Inference}

Mention that we are using the same inference method, but that we have expanded the set of local search operators (to accommodate change points and in response to observations).

\section{Translation of kernel functions}
\label{sec:translation}

The extended GPSS language produces sentences composed of (some number) base kernels (refer to a table) and the arbitrary application of addition, multiplication, change point and change window operators. 
In this section we recall some elementary facts about kernels and show how they reduce the problem of translation to simple or finite subproblems.

\paragraph{Changepoints}

We use equation \eqref{eq:cp} to represent changepoints (and similarly change windows) using sums, products, elementary kernels and univariate functions.

\paragraph{Distributivity}

Multiplication is distributive over addition for univariate and kernel functions so we can convert any kernel expression into a sum of products.

\paragraph{Additive kernels}

If $f_1(x) \dist \gp{}(0, \kernel_1)$ and $f_2(x) \dist \gp{}(0, \kernel_2)$ then $f_1(x) + f_2(x) \dist \gp{}(0, \kernel_1 + \kernel_2)$.
Therefore, a sum of kernels can be described as a sum of independent functions.
We now only need to describe arbitrary products of kernels.

\paragraph{Multiplication by univariate functions}

Suppose that $f(x) \dist \gp{}(0, \kernel)$ and $a : \mathcal{X} \to \mathcal{Y}$ is a deterministic univariate function.
Then $a(x)f(x) \dist \gp{}(0, g)$ where $g(x, x') = a(x)a(x')k(x, x')$.

The linear kernel, $\kLin$, has the form $\kLin(x, x') = c(x-a)(x'-a)$.
Setting $a(x) = \sqrt{c}(x-a)$ we see that multiplying a kernel, $\kernel$, by the linear kernel is equivalent to multiplying $\gp{}(0, k)$ by a linear function.
Similarly, changepoint and change window operators are equivalent to multiplying \gp{}s by sigmoids and hat functions respectively.

Therefore we can separate the descriptions of an arbitrary product into a deterministic univariate function envelope and a stationary kernel.
The univariate envelopes can be arbitrary products of linear functions and sigmoids which can be simply described by their parameters.

We now only need to describe arbitrary products of stationary kernels.

\paragraph{Redundancy}

At a syntactic level, $\kSE$ is idempotent \ie $\kSE \times \kSE = \kSE$.
For stationary kernels, $\kWN$ behaves like zero \ie $\kWN \times \kSE = \kWN \times \kPer =  \kWN$.
All stationary base kernels have a scale parameter; multiplying by the constant kernel $\kC$ is equivalent to changing this parameter.

We now only need to describe the base kernels and products of the form $\kSE \times \prod \kPer$.

\paragraph{Base kernels}

The priors induced by the base kernels are well understood, with simple descriptions of typical functions.

\begin{table}[ht]
\centering
\begin{tabular}{c|c}
Kernel & Function description \\
\midrule
$\kWN$ & White noise \\
$\kC$ & Constant \\
$\kLin$ & Linear \\
$\kSE$ & Smooth \\
$\kPer$ & Periodic \\
\end{tabular}
\label{table:base-kernels}
\end{table}

\paragraph{$\kSE \times \kPer$}

It was noted in \cite{DuvLloGroetal13} that this kernel gives rise to locally periodic functions.
Indeed, if the input space is restricted to a grid with spacing equal to the period of the periodic kernel then $\kSE \times \kPer = \kSE$.
We therefore see that the functional form of the periodicity varies like $\kSE$, giving rise to the local nature of the periodicity.

\paragraph{$\prod \kPer$}

Suppose that ${f_1(x) \dist \gp{}(0, \kernel_1)}$ and ${f_2(x) \dist \gp{}(0, \kernel_2)}$.
Then
\begin{equation}
{\textrm{Covar}(f_1(x)f_2(x), f_1(x')f_2(x')) = k_1(x,x')k_2(x,x')}
\end{equation}
\ie the product of functions gives rise to functions whose covariance structure is the same as the product of kernels.

\NA{
Therefore products of periodic kernels will look like products of periodic functions maybe.
Need to make this more exact.
}

\subsection{Ordering additive components}

An automatic regression system would ideally report the most interesting or important features of the data first.
However, measuring importance is subjective.
We make the choice that additive components are important if they produce accurate extrapolations an order the components by this metric (see an appendix for details).

\subsection{Things the automatic statistician says - put this in an appendix?}
\label{sec:example_translation}

Make a better version of this - perhaps describing some of the classical regression motifs.

\begin{table}[ht]
\begin{tabular}{l|l}
Kernel & Description \\
\midrule
$\kPer$ & An exactly periodic function \\
$\kPer \times \kSE$ & An approximately periodic function \\
$\kPer \times \kSE \times \kLin$ & An approximately periodic function with linearly varying amplitude \\
$\kLin$ & A linear function \\
$\kLin \times \kLin$ & A quadratic function \\
$\kPer \times \kLin \times \kLin$ & An exactly periodic function with quadratically varying amplitude\\
\end{tabular}
\caption{A demonstration of how it is possible to write an open-ended summarizer.  Different kernel structures modify the overall statistical structure of each component in independent ways.}
\label{table:descriptions}
\end{table}

\subsection{When is a description `correct'}

In the above we have described how to produce a description of the prior of a \gp{} described by an arbitrary kernel.
However, when describing a particular data set we are interested in the posterior of the function describing the data.

For example, when fitting a linear regression model, the fit to the data will always be linear, for any data set.
One can test the assumptions of linear regression by model checking.
For example, one can test the prior independence assumption of the residuals by computing statistics of the posterior residuals and comparing to those expected under the prior \ie comparing the prior and posterior of parts of the model.

In the same way, one must also check the models produced by GPSS; this can take the form of testing whether the posterior of each component resembles its prior as measured by a variety of statistics.
The plots in the later sections allow human model checking of this form, but ideally this process should be automated as well.
We are actively researching model checking methods for \gp{} models.

\NA{
Some thoughts\ldots
Optimising marginal likelihood is attempting to maximise the probability of a data set given the model.
The components can still be anti-correlated which is a problem - but model checking can see this if necessary.
A small amount of descriptions are about the posterior (\eg increasing / decreasing linear functions).
Posterior predictive checks compare posterior and prior (of the not parameters), so we believe our model is reasonable when prior and posterior align and we have a greater chance of this happening when we search over a large space of models.
What are consistency results about additive \gp{}s, or other additive nonparametric models - clearly there can be non-identifiability.
}

\section{Related work}

\paragraph{Equation learning}

In this work we have defined a search space of statistical models, where each model is characterised by a parametric covariance function.
Previous work has considered searching over parametric functions (cite equation learning).
Some functions are well explained parametrically \eg a logistic function, whereas others are best described by their covariance \eg an approximately periodic function.
It is our opinion that equation learning and covariance learning should ultimately be combined.

\paragraph{Searching over statistical models}

The original motivation for this line of work was the search over matrix factorisation models by (cite Roger).
This work and ours is possible since we can define a large space of models through a succinct langauge, inference can be performed for all sentences in this language and a basic search procedure is sufficient to find interesting models.
This is also true of - can we also cite some sort of graphical model work?

Defining infernce schemes that work for a very large class of models is also a research area - at the extreme end it is called probabilistic program (cite cite cite).

\paragraph{Natural-language descriptions of data}

Translation is not really done.
Perhaps cite Gowers as someone else who thinks communication is important.
Or perhaps grammar based understanding of natural scenes.

\section{Natural-language description of time series}
\label{sec:examples}

We demonstrate the ability of our procedure to describe a variety of patterns on two time series (full automatically produced reports for a greater number of data sets are provided as supplementary material).
The natural-language descriptions are clearly sensible, but since it is difficult to quantify the quality of a verbal description we also provide numerical evidence in support of our procedure in section~\ref{sec:numerical}.

%Our report generation procedure takes as its starting point a dataset and a composite kernel, which together define a joint posterior probability distribution over a sum of functions.
%The procedure summarizes properties of this complex distribution to the user through a comprehensive report.
%
%These reports are designed to be intelligible to non-experts, illustrating the assumptions made by the model, describing the model's posterior distribution, and most importantly, enabling model-checking.
%\subsection{Design goals}
%
%\begin{itemize}
%\item {\bf Intelligibility}
%The procedure was designed to produce reports intelligible to non-experts.  
%
%\item {\bf Illustrate Model Assumptions}
%One of the main design criteria when designing our procedure was to make clear what assumptions the model is making, and what these assumptions imply in terms of extrapolation.  Even when simple Gaussian process models are used, it is often unclear what structure was captured by the model and what was not.
%
%\item {\bf Illustrate Posterior Uncertainty}
%One of the selling points of Bayesian modeling over point estimates is that they produce a rich posterior distribution over possible explanations of the data.  However, this posterior distribution is often quite a complex object, and is usually difficult to summarize.
%
%\item {\bf Enable Model-checking}
%One of the most important reasons to examine a model is to discover flawed assumptions, or structure not captured by the model.  Simply examining residuals, cross-validation or marginal likelihood scores is often not sufficient to notice when the model is failing to capture.
%\end{itemize}
%
%\subsection{Report structure}

%These reports have three sections: an executive summary, a detailed discussion of each component, and a section discussing how the model extrapolates beyond the range of the data.

%\vspace{-0.05in}

\subsection{Summarizing 400 Years of Solar Activity}
\label{sec:solar}

We show excerpts from the report automatically generated on annual solar irradiation data from 1610 to 2011.
This time series has two interesting features: a roughly 11-year cycle of solar activity, and a period lasting from 1645 to 1715 with much smaller variance than the rest of the dataset.  This flat region corresponds to the Maunder minimum, a period in which sunspots were extremely rare \citep{lean1995reconstruction}.
%
The GPSS search procedure and automatic summary clearly identify these two features, as discussed below.

\paragraph{Executive Summary}

The first section of each report gives short summaries about each component in the model, and statistics describing the relative importance of the different components in explaining the data.

\begin{figure}[h]
\centering
\fbox{\includegraphics[trim=0cm 3.4cm 0cm 6.3cm, clip, width=0.98\columnwidth]{solarpages/02-solar-seperate-pages-2}}
\caption{
An example of an automatically-generated summary of a dataset.  The dataset is decomposed into diverse types of structures, and each structure is explained in simple terms.}
\label{fig:exec}
\end{figure}

The syntax of the kernels corresponding to the first four components are as follows
\begin{itemize}
  \item $\kC$
  \item $\kCW(\emptyset, \kC)$
  \item $\kCW(\kSE, \emptyset)$
  \item $\kCW(\kSE \times \kPer, \emptyset)$
\end{itemize}
where $\emptyset$ is the zero function.
Figure \ref{fig:exec} shows the automatically-generated summary of the solar dataset.
The short descriptions demonstrate how the kernel is split into univariate enveloping functions (from the change windows) and stationary kernels.
%
%
%The model uses 9 additive components to explain the data, and reports that the first 4 components explain more than 90\% of the variance in the data.
%This might seem incongruous with the observation that there are two main features of the data, but if we examine the first four components, we see that the first component is describing the mean of the dataset, the second is the Maunder minimum, the third describes the long-term trend, and the fourth describes the 11-year periodicity.
Just from the short summaries of the additive components we can see that the model has identified the Maunder minimum (second component) and 11-year solar cycle (fourth component).
%This might seem incongruous with the observation that there are two main features of the data, but if we examine the first four components, we see that the first component is describing the mean of the dataset, the second is the Maunder minimum, the third describes the long-term trend, and the fourth describes the 11-year periodicity.

%\subsubsection{Signal versus Noise}
%
%One design challenge we encountered was seperating the recovered structure into signal and noise.  Originally, the model always included a term corresponding to \iid{} additive Gaussian noise.  However, in practice, the distinction between signal and noise is unclear for two reasons.  First, a component which varies arbitrarily quickly in time can be indistinguishable from noise.  Second, the variance of the noise may change over time (called heteroskedasticity), and this sort of pattern may be considered part of the signal.
%Because of the blurry distinction between signal and noise, we include a table which summarizes the relative contribution of each component in terms of held-out predictive power.%  To do this, we order the components in terms of how much each one improves predictive performance in a 10-fold cross-validation procedure.  The intuition for this metric is that noise-like components do not contribute much to the extrapolation performance of the model, but that signal-like components do.
%
%
%\begin{figure}
%\centering
%\fbox{\includegraphics[width=0.98\columnwidth]{solarpages/02-solar-seperate-pages-3}}
%\caption{A table summarizing the relative contribution of the 9 different components of the model in terms of predictive performance.}
%\label{fig:table}
%\end{figure}
%
%Figure \ref{fig:table} show an example of this table on the solar dataset.

%Because the user may be interested in local or noisy components, we report all components to the user.  
%An interactive version of our procedure could allow users to specify which components are of interest, and group the remaining components into a single noise component.

\paragraph{Decomposition plots}

The second section of each report contains a detailed discussion of each component.
%Every component is plotted, and properties of the covariance structure are described.
%Some components are not meaningful when plotted on their own, so we also include plots of the cumulative sum of components.
%\paragraph{Automatic Plotting}
The posterior of the individual component and sum of all components so far is visualised by plots of the posterior mean and variance.
%First, the posterior mean and variance of each component is plotted on its own.
%Second, the posterior mean and variance of all components shown so far is plotted against the data.
%This progression of plots 
%By contrasting each of these plots with plots of earlier components, we can see 
%shows qualitatively how each component contributes to an overall explanation of the data.

%A second paragraph explains the improvement in predictive performance gained by including this component in the model. This is the same informatino as included in the executive summary.

\paragraph{Maunder minimum}

Figure \ref{fig:maunder} shows that GPSS has captured the unusual period of decreased solar activity from about 1645 to 1715 and is able to report this in natural language.

\begin{figure}[ht]
\centering
\fbox{\includegraphics[trim=0cm 0cm 0cm 0.7cm, clip, width=0.98\columnwidth]{solarpages/02-solar-seperate-pages-5}}
\caption{Discovering the Maunder minimum.  The kernel found by GPSS contained a pair of changepoints bracketing the period of low solar activity.}
\label{fig:maunder}
\end{figure}

\paragraph{Long term trend}

Having isolated the Maunder minimum, the model captures the long term trend of the rest of the data, shown in figure~\ref{fig:smooth}.

\begin{figure}[h!]
\centering
\fbox{\includegraphics[width=0.98\columnwidth]{solarpages/02-solar-seperate-pages-6}}
\caption{Characterizing the medium-term smoothness of solar activity levels.  By allowing other components to explain the periodicity, noise, and he Maunder minimum, we can isolate the part of the signal best explained by a slowly-varying trend.}
\label{fig:smooth}
\end{figure}

% is a good example of a meaningful component discovered by GPSS, whose meaning would be unclear without an individual plot.  


%In the history of solar activity, the Maunder minimum is a good example of a local change in covariance.  Specifically, 
%The changepoint kernels used by GPSS encode changes in covariance structure.
%For example, from about 1645 to 1715, solar activity decreased.
%, and very few sunspots were observed, a period called the Maunder Minimum \citep{lean1995reconstruction}.
%This feature was captured by the model by multiplying a constant kernel by two changepoint kernels.

\paragraph{Solar cycles}

Figure \ref{fig:periodic} shows that GPSS has isolated the approximately 11 year solar cycle.
By examining the parameters of the kernels comprising this component the description 

%with a pair of changepoint kernels.%shows exactly which sort of structure was recovered by this component.
%
%\begin{figure}[h!]
%\centering
%\fbox{\includegraphics[width=0.98\columnwidth]{solarpages/02-solar-seperate-pages-6}}
%\caption{Characterizing the medium-term smoothness of solar activity levels.  By allowing other components to explain the periodicity, noise, and the Maunder minimum, we can isolate the part of the signal best explained by a slowly-varying trend.}
%\label{fig:smooth}
%\end{figure}

%\paragraph{Isolating the smoothly-varying component} Examining the dataset by eye, overall solar activity seems to change slowly over decades.  However, this intuition seems difficult to formalize.  Linear or quadratic regression is clearly inappropriate, and methods based on local smoothing would need to control for the periodic component.  Luckily, the GPSS procedure does exactly this, allowing us to isolate the slowly-varying component of the data, without having to forecast either the Maunder minimum or the periodic variation.  Figure \ref{fig:smooth} shows the automatically-generated summary of this component.

\begin{figure}[ht]
\centering
\fbox{\includegraphics[trim=0cm 0cm 0cm 1.0cm, clip, width=0.98\columnwidth]{solarpages/02-solar-seperate-pages-7}}
\caption{Isolating the periodic component of the dataset.  By isolating this aspect of the statistical structure, we can easily observe additional features, such as the shape of the peaks and troughs, or the fact that the amplitude changes over time.}
\label{fig:periodic}
\end{figure}

%Figure \ref{fig:periodic} shows that GPSS has identified the approximately 11 year solar cycle.
%By isolating this component in separate plots it is easy to see the exact nature of the solar cycle \eg how the amplitude of this periodic component varies over time.
%This demonstrates one benefit of isolating individual components: we can now see, by eye, extra structure that was not explicitly captured by the model.  Specifically, we can see that the amplitude of the periodic component varies over time.

%and by comparing with figure \ref{fig:smooth}, we can see that it varies roughly in proportion to the overall magnitude of the signal.
%  This pattern suggests that some sort of log-transform might be appropriate for this dataset, or that the model should be extended in some way to capture this structure.

%\paragraph{Extrapolation plots}
%
%The third section of each report shows extrapolations into the future, as well as posterior samples from each individual component of the model.  These samples help to characterize the uncertainty expressed by the model, and the extent to which different components contribute to predicting the future behavior of a time series.
%%
%The predictive mean and variance of the signals shown in the summary plots are useful, but do not capture the joint correlation structure in the posterior.  Showing posterior samples is a simple and universal way to illustrate joint statistical structure.
%%
%\begin{figure}[ht]
%\centering
%\fbox{\includegraphics[trim=0cm 0cm 0cm 2.8cm, clip, width=0.98\columnwidth]{solarpages/02-solar-seperate-pages-13}}
%\caption{Sampling from the posterior.  These samples help show not just the predictive mean and variance, but also the predictive covariance structure.  Note, for example, that the predictive mean (left) does not exhibit periodicity, but the samples (right) do.}
%\label{fig:extrap-full}
%\end{figure}
%%
%For example,
%%  shows the predictive mean and variance given the entire model. 
%it is not clear from the left-hand plot in figure \ref{fig:extrap-full} whether or not the periodicity of the dataset is expected to continue into the future.  However, from the samples on the right-hand size, we can see that this is indeed the case.  

%\begin{figure}[h!]
%\centering
%\fbox{\includegraphics[width=0.98\columnwidth]{solarpages/02-solar-seperate-pages-16}}
%\caption{Extrapolating a single component of the model.  Because our model class allows us to isolate individual components, we can show the extent to which the model expects different trends to continue.  We also observe that the posterior samples are quite different from the posterior mean, away from the data.}
%\label{fig:extrap-smooth}
%\end{figure}

%\paragraph{Extrapolating individual components}
%We can also examine the model's expectations about the future behavior of individual components through sampling.  Further plots in the extrapolation section show posterior samples for each individual additive component. %For example, in figure \ref{fig:extrap-smooth}, we are shown samples of the predictive distribution of the smooth component of variation.  This plot indicates that the model considers a wide range of future average intensities to be plausible, but that it always expects those average intensities to vary smoothly.

%\section{Related Work}

%There exists a vast literature on both model visualization and model checking.


%\paragraph{Structure learning in Bayesian networks}
%Similar idea of discovering semantics via model search.
%Semantics are more vague though \ie a probability table is not an entirely concise summary

%\paragraph{Linear model}
%These discover highly interpretable semantics but are limited in expressivity

%\paragraph{Nonparametric additive models}
%Highly flexible but semantics are vague \ie can only talk about smooth functions

%\paragraph{Equation learning}
%Very flexible but semantics of equations do not map onto human understanding \eg saw tooth vs Fourier decomposition of a saw tooth - which is more human understandable?
%How would you explain a sensor error with Eureqa style equations.

%\paragraph{Deep learning}
%Again very flexible but the semantics are not usually human interpretable.
%How can we understand the output of complex representation learning algorithms without human intervention (\eg recognising that your deep net has become a cat classifier).

%\paragraph{Kernel search}
%Can use the precise semantics of linear models or the vague semantics of nonparametric additive models and other components along this spectrum.
%Flexible modelling with components that a human might use to describe what is going on.

\subsection{Describing heteroskedasticity}
\label{sec:airline}

Give the same treatment to the airline data set.

\subsection{Straw men comparisons}

Nothing really quite compares to this work, the closest would be equation learning and trend-periodicity-short-term models.
Easy to show that equation learning can fit data but does not provide neat descriptions interpretable at a human level.

\section{Numerical evaluation}
\label{sec:numerical}

Can try interpolation and extrapolation (maybe just extrapolation since it is the harder task - explain how the various algorithms would perform at interpolation).
Can compare to SE, equation learning, Spectral, MKL, SE + change points, Trend + period + short-term (we can do all of these with current code - except for equation learning).
Mention that the data sets are chosen to highlight what features can be modelled by various techniques and which cannot rather than the traditional, my model rules the world.

Should probably compare to ARIMA class models - they will probably do fine for stationarity.
Or am I opening a huge can of worms here - should the actual comparison be with structure search for time series models?

%\subsection{Deficiencies of the modelling language}

%Show humility with another failure mode (\eg variable period).

\section{Discussion}

We should demand the same transparency of statistical models as we do a junior employee - although continuing the analogy we sometimes do not expect transparency from a genius (\ie neural nets?).

If we can explain something in plain language then we should.

What are the correct metrics?
A good description should reflect a statistical model that can predict well.
Interpolation and extrapolation are one measure of prediction.
Another measure could be scientific discovery - the feels like the ultimate test.

\paragraph{Source Code}
Python code to perform all experiments is available on github.\footnote{Available at 
\href{http://www.github.com/jamesrobertlloyd/gpss-research}
{\texttt{github.com/jamesrobertlloyd/gpss-research}}}
%All \gp{} hyperparameter tuning was performed by automated calls to the GPML toolbox\footnote{Available at 
%\href{http://www.gaussianprocess.org/gpml/code/}
%{\texttt{www.gaussianprocess.org/gpml/code/}}
%}

%\section{Discussion}

%\begin{quotation}
%``The availability of 'user-friendly' statistical software has caused authors to become increasingly careless about the logic of interpreting their results, and to rely uncritically on computer output, often using the 'default option' when something a little different (usually, but not always, a little more complicated) is correct, or at least more appropriate.''
% In trying to practice this art, the Bayesian has the advantage because his formal apparatus already developed gives him a clearer picture of what to expect, and therefore a sharper perception for recognizing the unexpected.

%\defcitealias{dyke1997avoid}{G. Dyke, 1997}
%\hspace*{\fill}\citet{Jaynes85highlyinformative}
%\hspace*{\fill}\citetalias{dyke1997avoid}
%\end{quotation}

%In this paper, we exhibited the output of a method for automatically constructing and summarizing a compositional Gaussian process regression model in natural language.
%These summaries can enable human experts and non-experts to understand the implications of a model, check its plausibility, and notice structure not yet captured by the model.

\bibliography{gpss}
\bibliographystyle{format/icml2014}

\end{document} 
