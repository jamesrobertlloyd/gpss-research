\input{texhead}

\graphicspath{{/Users/zoubin/gatsby/papers/nips00/}{/Users/zoubin/gatsby/talks/PSFILES/}{/Users/zoubin/gatsby/talks/cmu02inf/forzoubin/}{/Users/zoubin/gatsby/talks/aistats03/}{/Users/zoubin/gatsby/course03/COURSEFIGS/}{/Users/zoubin/gatsby/talks/icml04tutorial/}{/Users/zoubin/gatsby/talks/gatsby04/}{/Users/zoubin/gatsby/talks/tubingen05/}{/Users/zoubin/gatsby/talks/cmu06mld-lunch/}{/Users/zoubin/gatsby/talks/valencia06/}{/Users/zoubin/gatsby/talks/cavendish07/}{/Users/zoubin/gatsby/talks/nips05ed/}{/Users/zoubin/gatsby/talks/uai05tutorial/}{/Users/zoubin/gatsby/talks/mlsp10/}{/Users/zoubin/gatsby/talks/oxford13/}}

% \usepackage{soul}
\usepackage[display,verbose]{/Users/zoubin/gatsby/tex/inputs/texpower}
\usepackage{/Users/zoubin/gatsby/tex/inputs/msover}

\usepackage{algorithmic}
\usepackage[tight]{subfigure}
% \usepackage{bbm}

\newcommand{\bZ}{\mathbf{Z}}
\newcommand{\bX}{\mathbf{X}}
\newcommand{\D}{\mathcal{D}}
\newcommand{\B}{\mathcal{B}}
\newcommand{\R}{\Re}
\newcommand{\N}{\mathcal{N}}
\newcommand{\deff}{\stackrel{\mathrm{def}}{=}}
\newcommand{\KL}{\mathrm{KL}}
\newcommand{\qi}{q_{{\scriptscriptstyle\backslash} \!i}(\btheta)}
% \newcommand{\balpha}{\boldsymbol{\alpha}}
\newcommand{\bt}{\mathbf{t}}
\newcommand{\bff}{\mathbf{f}}
%\newcommand{\ff}{\mathbf{f}}
\newcommand{\bfb}{\bar{\bff}}
\newcommand{\bK}{\mathbf{K}}
\newcommand{\rd}{\mathrm{d}}
\newcommand{\cN}{\mathcal{N}}
\newcommand{\bzero}{\mathbf{0}}
\newcommand{\bKC}[1]{\mathbf{K}_{\scriptscriptstyle #1}}
\newcommand{\bKi}[2]{\mathbf{K}_{\scriptscriptstyle #1}^{\scriptscriptstyle #2}}
\newcommand{\bLa}{\mathbf{\Lambda}}
\newcommand{\cO}{\mathcal{O}}
\newcommand{\bXb}{\bar{\bX}}
\newcommand{\figwid}{2.8in}
\newcommand{\negs}{-3ex}


\newlength{\oin}
\setlength{\oin}{0.9in} % 0.45 for slides


\begin{document}
\titleslide{The Automatic Statistician}{\PineGreen{James Lloyd, David
  Duvenaud (Cambridge) \\ and Roger Grosse, Josh Tenenbaum
  (MIT)}}{Imperial College Lectures, 2014}  

%%%%%%%%%%%%%%%%%%%%%%%%%%%%%%%%%%%%%%%%%%%%%%%%%%%%%%%%%%%%%%%%%%%%%%
\foilhead{Motivation}

\begin{itemize}
\item We live in an era of {\bf abundant data}
\item Scientific, commercial and societal uses of this data drive the
  need for exploratory data analysis and
  prediction methods, but there are {\bf too few experts statisticians and
  data scientists} to provide these services.
\item Many aspects of statistical inference can be automated, and one
  of the goals of machine learning and artificial intelligence is to
  develop powerful tools for understanding data that require minimal
  expert input. 
\item By trying to build an ``Automatic Statistician'' we can
\begin{itemize}
\item provide a set of useful tools for understanding certain kinds of
  data
\item uncover challenging research problems in automated inference,
  model construction and comparison, and data visualisation and
  interpretation 
\item advance the field of machine learning in general
\end{itemize}

\end{itemize}

%%%%%%%%%%%%%%%%%%%%%%%%%%%%%%%%%%%%%%%%%%%%%%%%%%%%%%%%%%%%%%%%%%%%%%
\foilhead{Ingredients}

\begin{itemize}
\item Probabilistic modelling
\item Model selection and marginal likelihoods
\item Bayesian nonparametrics
\item Gaussian processes
\item Change-point kernels
\end{itemize}

%%%%%%%%%%%%%%%%%%%%%%%%%%%%%%%%%%%%%%%%%%%%%%%%%%%%%%%%%%%%%%%%%%%%%%
\foilhead{Probabilistic Modelling}

\parbox{0.95\textwidth}{
\large
\begin{itemize}
\item A model describes data that one could observe from a system
\item If we use the mathematics of probability theory to express all
  forms of uncertainty and noise associated with our model...
\item ...then {\it inverse probability} (i.e.\ Bayes rule) allows us
  to infer unknown quantities, adapt our models, make predictions and
  learn from data.
\end{itemize}
}

%%%%%%%%%%%%%%%%%%%%%%%%%%%%%%%%%%%%%%%%%%%%%%%%%%%%%%%%%%%%%%%%%%%%%%
\foilhead{Bayesian Machine Learning} 
% from ICML tutorial...

\begin{center}
\fbox{\parbox{5.5in}{
{\it Everything follows from two simple rules:}\\[0.5ex]
\Red{\bf Sum rule:} \hspace{7ex} $P(x) = \sum_y P(x,y)$ \\[0.2ex]
\Red{\bf Product rule:} \hspace{1ex} $P(x,y) = P(x)P(y|x)$
}} 
\end{center}


\vspace{2ex}

\parbox{3.8in}{
\[
\PineGreen{P(\theta|\D,m)} = \frac{P(\D|\theta,m) \Brown{P(\theta|m)}}{\Orange{P(\D|m)}}
\]
} \hspace{2ex}
\parbox{5in}{\small
\begin{center}
\begin{tabular}{ll}
$P(\D|\theta,m)$ & likelihood of parameters $\theta$ in model $m$ \\
\Brown{$P(\theta|m)$} & prior probability of $\theta$ \\
\PineGreen{$P(\theta|\D,m)$} & posterior of $\theta$ given data $\D$
\end{tabular}
\end{center}
}
\vspace*{1ex}

\Blue{\bf Prediction:}
\vspace*{-2ex}
\begin{eqnarray*}
P(x|\D,m)  &= & \int P(x|\theta,\D,m) \PineGreen{P(\theta|\D,m)} d \theta\\
% P(x|\D,m)  &= & \int P(x|\theta) P(\theta|\D,m) d \theta
% \mbox{\hspace*{3ex} (for many models)}\\
\end{eqnarray*}

\Blue{\bf Model Comparison:}
\vspace*{-1ex}
\begin{eqnarray*}
P(m|\D) & = & \frac{\Orange{P(\D|m)}P(m)}{P(\D)}\\
\Orange{P(\D|m)} & = & \int P(\D|\theta,m) \Brown{P(\theta|m)}\, d \theta 
\end{eqnarray*}

%%%%%%%%%%%%%%%%%%%%%%%%%%%%%%%%%%%%%%%%%%%%%%%%%%%%%%%%%%%%%%%%%%
\foilhead{Model Comparison}

\centerline{\includegraphics[width=7\oin]{ofitnew}} 


%%%%%%%%%%%%%%%%%%%%%%%%%%%%%%%%%%%%%%%%%%%%%%%%%%%%%%%%%%%%%%%%%%
\foilhead{Bayesian Occam's Razor and Model Comparison}
\vspace*{-0.3in}
Compare model classes, e.g.\ $m$ and $m'$, using posterior
probabilities given $\D$: 
\vspace*{-0.2in}
\[
p(m|\D) = \frac{\Red{p(\D|m)} \, p(m)}{p(\D)}, \hspace*{0.5in}
\Red{p(\D|m)} = \int p(\D|\btheta, m) \;
p(\btheta|m) \; d {\btheta}
\]
{\bf Interpretations of the \Red{Marginal Likelihood (``model
    evidence'')}:} 
{\small
\begin{itemize}
\item The probability that {\em randomly selected}\ parameters from the
prior would generate $\D$. 
\item Probability of the data under the model, {\em averaging} over all
  possible parameter values. 
\item $\log_2 \left( \frac{1}{\Red{p(\D|m)}} \right)$ is the number of
  {\em bits of surprise} at observing data $\D$ under model $m$.
\end{itemize}
}

\parbox{5.5in}{
Model classes that are \PineGreen{too simple} are unlikely to
generate the data set. \\[-0.5ex]

Model classes that are \Cyan{too complex} can generate many possible
data sets, so again, they are unlikely to generate that particular
data set at random.} \hspace{2ex}
\parbox{4.5in}{
\centerline{\includegraphics[width=4.2in]{ockham2}}
}


%%%%%%%%%%%%%%%%%%%%%%%%%%%%%%%%%%%%%%%%%%%%%%%%%%%%%%%%%%%%%%%%%%
\foilhead{Bayesian Model Comparison: \normalsize Occam's Razor at Work }

\vspace{0.1\oin}
\centerline{\includegraphics[width=5\oin,angle=90]{ofitsamps2} 
\raisebox{36pt}[0pt][36pt]{
\includegraphics[width=4\oin,angle=90]{ofitpost2}}}

\vspace*{-2ex}
For example, for quadratic polynomials ($m=2$): $ y = a_0 + a_1 x + a_2
x^2 + \epsilon$, where $\epsilon \sim {\cal   N}(0,\sigma^2)$ and parameters $\btheta = (a_0 \; a_1
\; a_2 \; \sigma)$ \\[1ex]
{\tt demo: polybayes}\\
% {\tt demo: run\_simple}


%%%%%%%%%%%%%%%%%%%%%%%%%%%%%%%%%%%%%%%%%%%%%%%%%%%%%%%%%%%%%%%%%%%%%%
\foilhead{Parametric vs Nonparametric Models}

\begin{itemize}
\hrule

\item \Red{\em Parametric models} assume some \Red{finite set of
  parameters} $\theta$. Given the parameters, future predictions, $x$, are
  independent of the observed data, $\D$:
\[
P(x|\theta,\D) = P(x|\theta)
\]
therefore $\theta$ capture everything there is
to know about the data.  
\item So the complexity of the model is bounded even
if the amount of data is unbounded. This makes them not very
flexible. \\[1ex] 

\hrule

\item \Blue{\em Non-parametric models} assume that the data distribution
  cannot be defined in terms of such a finite set of parameters. But they
  can often be defined by assuming an \Blue{\em infinite
  dimensional} $\theta$. Usually we think of $\theta$ as a \Blue{\em function}.
\item The amount of information that $\theta$ can capture about the data
  $\D$ can grow as the amount of data grows. This makes them more flexible.
\end{itemize}

\hrule

%%%%%%%%%%%%%%%%%%%%%%%%%%%%%%%%%%%%%%%%%%%%%%%%%%%%%%%%%%%%%%%%%%%%%%
\foilhead{Bayesian nonparametrics}

{\em A simple framework for modelling complex data.} \\

{\em Nonparametric models can be viewed as having
  infinitely many parameters} \\

Examples of non-parametric models: \\

\begin{tabular}{lll} \hline
Parametric & Non-parametric & Application \\ \hline
polynomial regression & Gaussian processes & function approx. \\
logistic regression & Gaussian process classifiers & classification \\
mixture models, k-means & Dirichlet process mixtures &
clustering \\
hidden Markov models & infinite HMMs & time series \\
factor analysis / pPCA / PMF & infinite latent factor models &
feature discovery \\
... & & \\
\hline
\end{tabular}

%%%%%%%%%
\foilhead{Nonlinear regression and Gaussian processes}
\vspace*{-2ex}
Consider the problem of \Green{nonlinear regression}:

You want to learn a \Blue{function $f$} with \Red{error bars} from
\Magenta{data $\D = \{\bX,\by\}$} \\

\begin{center}
\includegraphics{prednotitle}
\end{center}

A \Green{Gaussian process} defines a distribution over functions
$p(f)$ which can be used for Bayesian regression:
\begin{equation*}
  p(f|\D) = \frac{p(f) p(\D|f)}{p(\D)}
\end{equation*}

Let $\ff = (f(x_1), f(x_2), \ldots, f(x_n))$ be an $n$-dimensional
vector of function values evaluated at $n$ points $x_i \in {\cal X}$.
Note, $\ff$ is a random variable. \\

{\bf Definition:} $p(f)$ is a \Red{Gaussian process} if for {\em any} 
finite subset $\{x_1, \ldots, x_n \} \subset {\cal X}$, the marginal 
distribution over that subset $p(\ff)$ is multivariate
Gaussian. 

%%%%%%%%%
\foilhead{A picture}

\centerline{\includegraphics[width=6.5in]{gpc-cube}}


%%%%%%%%%%%%%%%%%%%%%%%%%%%%%%%%%%%%%%%%%%%%%%%%%%%%%%%%%%%%%%%%%%%%%%
\foilhead{Gaussian process covariance functions (kernels)}

$p(f)$ is a \Red{Gaussian process} if for {\em any} 
finite subset $\{x_1, \ldots, x_n \} \subset {\cal X}$, the marginal 
distribution over that finite subset $p(\ff)$ has a multivariate
Gaussian distribution. \\

Gaussian processes (GPs) are parameterized by a \Blue{mean function,
$\mu(x)$}, and a \Green{covariance function, or kernel, $K(x,x')$}.
\vspace*{-1ex} 
\[
p(f(x),f(x')) = \mbox{N}(\Blue{\bmu},\Green{\Sigma})
\]
where % \vspace*{-2ex}
\[
\Blue{\bmu= \left [ \begin{array}{c} \mu(x) \\ \mu(x') \end{array}
\right ] }
\hspace{2ex} 
\Green{
\Sigma = \left [ \begin{array}{cc} K(x,x) & K(x,x') \\
K(x',x) & K(x',x') \end{array}  
\right]}
\] \\
and similarly for $p(f(x_1), \ldots, f(x_n))$ where now $\bmu$ is an
$n \times 1$ vector and $\Sigma$ is an $n \times n$ matrix.\\


%%%%%%%%%%%%%%%%%%%%%%%%%%%%%%%%%%%%%%%%%%%%%%%%%%%%%%%%%%%%%%%%%%%%%%
\foilhead{Gaussian process covariance functions}

Gaussian processes (GPs) are parameterized by a \Blue{mean function,
$\mu(x)$}, and a \Green{covariance function, $K(x,x')$}, where 
$\mu(x) = \mbox{E}(f(x))$ and $K(x, x') =
\mbox{Cov}(f(x),f(x'))$.\\

An example covariance function: 
\[
\Green{K(x,x') = v_0 \exp \left\{ -
\left(\frac{|x-x'|}{r}\right)^\alpha \right\} + v_1 + v_2 \,
\delta_{ij}} 
\]
with parameters $(v_0, v_1, v_2, r, \alpha)$. 
\\

\parbox{0.6\textwidth}{These kernel parameters are \Red{interpretable}\\and can be learned
from data:} 

\vspace*{-8ex}
\phantom{.} \hfill
\begin{tabular}{|ll|} \hline
\Green{$v_0$} & signal variance \\
\Green{$v_1$} & variance of bias \\
\Green{$v_2$} & noise variance \\
\Green{$r$} & lengthscale \\
\Green{$\alpha$} & roughness \\ \hline
\end{tabular}

\vspace*{2ex}
Once the mean and covariance functions are defined, everything else
about GPs follows from the basic rules of probability
applied to mutivariate Gaussians.


%%%%%%%%%%%%%%%%%%%%%%%%%%%%%%%%%%%%%%%%%%%%%%%%%%%%%%%%%%%%%%%%%%%%%%
\foilhead{Samples from GPs with different $K(x,x')$}

% \newcommand{\figwid}{2.8in}
% \newcommand{\negs}{-3ex}

\includegraphics[width=\figwid]{figs/samp1a} \hspace*{\negs}
\includegraphics[width=\figwid]{figs/samp1b} \hspace*{\negs}
\includegraphics[width=\figwid]{figs/samp1c} \hspace*{\negs}
\includegraphics[width=\figwid]{figs/samp1d}

\vspace*{-1.5ex}
\includegraphics[width=\figwid]{figs/samp4a} \hspace*{\negs}
\includegraphics[width=\figwid]{figs/samp4b} \hspace*{\negs}
\includegraphics[width=\figwid]{figs/samp4c} \hspace*{\negs}
\includegraphics[width=\figwid]{figs/samp4d}

\vspace*{-1.5ex}
\includegraphics[width=\figwid]{figs/samp10a} \hspace*{\negs}
\includegraphics[width=\figwid]{figs/samp10b} \hspace*{\negs}
\includegraphics[width=\figwid]{figs/samp10c} \hspace*{\negs}
\includegraphics[width=\figwid]{figs/samp10d}

\vspace*{-1ex}
{\tt gpdemogen}


%%%%%%%%%%%%%%%%%%%%%%%%%%%%%%%%%%%%%%%%%%%%%%%%%%%%%%%%%%%%%%%%%%%%%%
\foilhead{Prediction using GPs with different $K(x,x')$}

A sample from the prior for each covariance function:

\includegraphics[width=\figwid]{figs/samp1a} \hspace*{\negs}
\includegraphics[width=\figwid]{figs/samp1b} \hspace*{\negs}
\includegraphics[width=\figwid]{figs/samp1c} \hspace*{\negs}
\includegraphics[width=\figwid]{figs/samp1dd}

\vspace{1ex}

Corresponding predictions, mean with two standard deviations:

\includegraphics[width=\figwid]{figs/preda} \hspace*{\negs}
\includegraphics[width=\figwid]{figs/predb} \hspace*{\negs}
\includegraphics[width=\figwid]{figs/predc} \hspace*{\negs}
\includegraphics[width=\figwid]{figs/predd}

\vspace*{-1ex}
{\tt gpdemo}

%%%%%%%%%%
\foilhead{Change Point Kernels}

Assume $\Redd{f_1(x)} \sim GP(0,k_1)$ and $\Blue{f_2(x)} \sim GP(0,k_2)$. Define:
\[
f(x) = (1-\sigma(x)) \Redd{f_1(x)} + \sigma(x) \Blue{f_2(x)}
\]

\parbox{0.6\textwidth}{where $\sigma$ is a sigmoid function between 0 and 1, such as the
logistic function: $\sigma(x) = 1/(1+\exp(-x))$}
\parbox{0.4\textwidth}{\centerline{\includegraphics[width=0.25\textwidth]{sigm}}}

Then $f \sim GP(0,k)$, where
\[
k(x,x') = (1-\sigma(x)) \, \Redd{k_1(x,x')}  \, (1-\sigma(x')) + \sigma(x) \,
\Blue{k_2(x,x')} \, \sigma(x') 
\]

We can parametrise the location $\tau$ and abruptness $a$ of the changepoint by replacing
$\sigma(x)$  with $\sigma(a(x-\tau))$. \\

{\it Intutively (in one-dimension), the function $f$ behaves like
$\Redd{f_1}$ before $\tau$ and like $\Blue{f_2}$ after $\tau$. }

\centerline{\PineGreen{\small (cf. Garnett, Osborne and Roberts, 2009)}}

%%%%%%%%%%
\foilhead{Change Windows}

A {\bf change window} (or interval) is simply defined as a changepoint from $\Redd{f_1(x)}$ to
$\Blue{f_2(x)}$ followed by a changepoint back to $\Redd{f_1(x)}$.\\

We can represent this via a product of two sigmoids with different offsets:

\[
f(x)  = (1-(1-\sigma(a(x-\tau_2))) \sigma(a(x-\tau_1))) \Redd{f_1(x)}  + 
(1-\sigma(a(x-\tau_2))) \sigma(a(x-\tau_1)) \Blue{f_2(x)} 
\]

This looks a bit messy but it just smoothly switches on $\Blue{f_2}$
between $\tau_1$ and $\tau_2$.

\centerline{\includegraphics[width=0.5\textwidth]{maunder-data}}

\centerline{\parbox{0.65\textwidth}{\small Solar irradiance data form 1600s showing the Maunder minimum where
sunspot activity was extremely rare.}}
%%%%%%%%%%
\foilhead{The Automatic Statistician}


%%%%%%%%%%
\foilhead{How do we learn the kernel?}

\begin{itemize}

\item {\bf Usual approach:}  parametrise the kernel with a few
  hyperparameters and optimise or infer these. An example covariance function: 
\[
\Green{K(x,x') = v_0 \exp \left\{ -
\left(\frac{|x-x'|}{r}\right)^\alpha \right\} + v_1 + v_2 \,
\delta_{ij}} 
\]
with parameters $(v_0, v_1, v_2, r, \alpha)$. 

\item {\bf Our approach:} Define a grammar over kernels and search over
  this grammar to discover an appropriate and interpretable structure
  of the kernel.

% We call our method: Automatic Bayesian Covariance Discovery (ABCD)

\end{itemize}


%%%%%%%%%%%%%%%%%%%%%%%%%%%%%%%%%%%%%%%%%%%%%%%%%%%%%%%%%%%%%%%%%%%%%%
\foilhead{Kernel Composition}

By taking a few simple {\bf base kernels} and two {\bf composition rules},
{\em kernel addition and multiplication}, we can span a rich space of
structured kernels.

\begin{tabular}{cc}
\includegraphics[width=0.5\textwidth]{gp-comp-basekernels-b} & 
\includegraphics[width=0.4\textwidth]{gp-comp-composition-b}
\end{tabular}

\centerline{\PineGreen{\small (w/ Duvenaud, Lloyd, Grosse, and Tenenbaum, ICML 2013)}}
\centerline{\PineGreen{\small see also (Wilson and Adams, ICML 2013)}}


%%%%%%%%%%%%%%%%%%%%%%%%%%%%%%%%%%%%%%%%%%%%%%%%%%%%%%%%%%%%%%%%%%%%%%
\foilhead{Kernel Composition}

\begin{center}
\begin{tabular}{cc}
\includegraphics[width=0.3\textwidth]{autostat-base-kernels} & 
\includegraphics[width=0.45\textwidth]{autostat-kernel-motifs}
\end{tabular}
\end{center}

Different kernels express a variety of covariance
structures, such as local smoothness or periodicity. New
kernels can be constructed by taking the product of a set of
base kernels to express richer structures, (e.g. locally periodic, or
heteroscedastic) 

\begin{itemize}
\item Search starts with the {\bf base kernels} for the GP, and
  applies different operations (addition, multiplication, CP, CW) to
  explore kernels spanned by the grammar.

\item For efficiency, kernel hyperparameters are optimised rather than
integrated out

\item Each resulting model is scored using the {\bf marginal
likelihood} penalised by a BIC term for number of
hyperparameters. 
\end{itemize}

%%%%%%%%%%%%%%%%%%%%%%%%%%%%%%%%%%%%%%%%%%%%%%%%%%%%%%%%%%%%%%%%%%%%%%
\foilhead{Kernel Composition: Mauna Loa CO$_2$ Keeling Curve}

% \centerline{\includegraphics[width=\textwidth]{gp-comp-mauna-loa-extrap}}
\centerline{\includegraphics[width=\textwidth]{mauna-search1c}}

%%%%%%%%%%%%%%%%%%%%%%%%%%%%%%%%%%%%%%%%%%%%%%%%%%%%%%%%%%%%%%%%%%%%%%
\foilhead{Kernel Composition: Mauna Loa CO$_2$ Keeling Curve}

% \centerline{\includegraphics[width=\textwidth]{gp-comp-mauna-loa-extrap}}
\centerline{\includegraphics[width=\textwidth]{mauna-search2c}}

%%%%%%%%%%%%%%%%%%%%%%%%%%%%%%%%%%%%%%%%%%%%%%%%%%%%%%%%%%%%%%%%%%%%%%
\foilhead{Kernel Composition: Mauna Loa CO$_2$ Keeling Curve}

% \centerline{\includegraphics[width=\textwidth]{gp-comp-mauna-loa-extrap}}
\centerline{\includegraphics[width=\textwidth]{mauna-search3c}}

%%%%%%%%%%%%%%%%%%%%%%%%%%%%%%%%%%%%%%%%%%%%%%%%%%%%%%%%%%%%%%%%%%%%%%
\foilhead{Kernel Composition: Mauna Loa CO$_2$ Keeling Curve}

% \centerline{\includegraphics[width=\textwidth]{gp-comp-mauna-loa-extrap}}
\centerline{\includegraphics[width=\textwidth]{mauna-search4c}}

%%%%%%%%%%%%%%%%%%%%%%%%%%%%%%%%%%%%%%%%%%%%%%%%%%%%%%%%%%%%%%%%%%%%%%
\foilhead{Example: An automatic analysis}

\includegraphics[width=\textwidth]{airline-analysis-c} 

%%%%%%%%%%%%%%%%%%%%%%%%%%%%%%%%%%%%%%%%%%%%%%%%%%%%%%%%%%%%%%%%%%%%%%
% \foilhead{Kernel Composition: Airline passenger prediction}

% \vspace*{-2ex}
% \centerline{\includegraphics[width=0.3\textwidth]{gp-comp-airline-decomp}}


%%%%%%%%%%%%%%%%%%%%%%%%%%%%%%%%%%%%%%%%%%%%%%%%%%%%%%%%%%%%%%%%%%%%%%
\foilhead{Kernel Composition: predictive results}

\centerline{\includegraphics[width=\textwidth]{gp-comp-table}}

%%%%%%%%%%%%%%%%%%%%%%%%%%%%%%%%%%%%%%%%%%%%%%%%%%%%%%%%%%%%%%%%%%%%%%
\foilhead{Predictive Results}

\centerline{\includegraphics[width=0.55\textwidth]{autostat-rmse-results}}

Combined results over 13 time series data sets comparing 5 methods\\

{\small
\begin{tabular}{ll}
GPSS: & Automatic Statistician using Gaussian process structure search\\
TCI: & trend-cyclical-irregular models (e.g.\ Lind et al., 2006) \\
SP: & spectral kernels (Wilson \& Adams, 2013) \\
SE: & additive nonparametric regression (SE) (e.g. Buja et al.,
1989)\\ 
CP: & change point modelling (CP) / multi resolution GP
(e.g. Garnett et al., 2009) \\
EL: & equation learning  using Eureqa (Nutonian, 2011) \\
\end{tabular}
}


%%%%%%%%%%%%%%%%%%%%%%%%%%%%%%%%%%%%%%%%%%%%%%%%%%%%%%%%%%%%%%%%%%
\foilhead{Distributivity helps interpretability}
\centerline{\includegraphics[width=\textwidth]{airline-full-analysis-c}}

%%%%%%%%%%%%%%%%%%%%%%%%%%%%%%%%%%%%%%%%%%%%%%%%%%%%%%%%%%%%%%%%%%
\foilhead{Generating text output from the Automatic Statistician}

\centerline{\includegraphics[width=0.6\textwidth]{autostat-text1}}

\centerline{\includegraphics[width=0.6\textwidth]{autostat-text2}}

{\small ``Automatic Construction and Natural-language Description of Additive
Nonparametric Models''} \\
\PineGreen{\small (w/ James Lloyd, David Duvenaud,
  Roger Grosse and Josh Tenenbaum, NIPS workshop, 2013 )}
%%%%%%%%%%%%%%%%%%%%%%%%%%%%%%%%%%%%%%%%%%%%%%%%%%%%%%%%%%%%%%%%%%
\foilhead{Example reports}

\begin{center}
{\tt 01-airline.pdf \\
02-solar.pdf \\
07-call-centre.pdf \\
09-gas-production.pdf\\
}
\end{center}

%%%%%%%%%%%%%%%%%%%%%%%%%%%%%%%%%%%%%%%%%%%%%%%%%%%%%%%%%%%%%%%%%%
\foilhead{Challenges}

\begin{itemize}
\item Trading off predictive performance and interpretability
\item Expressing a large and flexible enough class of models so that
  different kinds of data can be captured
\item The computational complexity of searching a huge space of models
\item Translating complex modelling constructs into the English language;
  automatically generating relevant visualisations
\end{itemize}


%%%%%%%%%%%%%%%%%%%%%%%%%%%%%%%%%%%%%%%%%%%%%%%%%%%%%%%%%%%%%%%%%%
\foilhead{Current and Future Directions}

\begin{itemize}
\item Automatic Statistician for:
\begin{itemize}
\item multivariate nonlinear regression $y = f(\bx)$
\item classification
\item completing and interpreting tables and databases
\end{itemize}
\item Probabilistic Programming
\begin{itemize}
\item probabilistic models are expressed in a general (Turing complete)
  programming language (e.g. Church/Venture/Anglican)
\item a {\bf universal inference engine} can then be used to infer
  unobserved variables given observed data
\item this can be used to implement seach over the model space in an
  automated statistician
\end{itemize}

\end{itemize}



%%%%%%%%%%%%%%%%%%%%%%%%%%%%%%%%%%%%%%%%%%%%%%%%%%%%%%%%%%%%%%%%%%
\foilhead{Summary}

\vspace*{-3ex}
\begin{itemize}
% {\large
\item The Automatic Statistician project aims to automate certain
  kinds of exploratory and predictive modelling
\item Conceptually, we follow a Bayesian framework, relying in
  particular on Bayesian nonparametric models for flexibility 
\item The ultimate aim is to produce output that is interpretable by a
  reasonably numerate non-statistician
\item We have a system that can produce readable 10-15 page reports from one
  dimensional time series, capturing non-stationarity, change-points
  and change-windows, periodicity, trends, etc 
\item Predictive performance seems very competitive 
\end{itemize}

% \begin{minipage}[t!]{0.7\textwidth}
% \phantom{.}

%\centerline{\tt http://learning.eng.cam.ac.uk/zoubin} 
% \centerline{\tt zoubin@eng.cam.ac.uk}
% \end{minipage}
% \begin{minipage}[t!]{0.3\textwidth}
% \hspace*{0.7\textwidth} \includegraphics[width=0.2\textwidth]{indian2}
% \end{minipage}

%%%%%%%%%%%%%%%%%%%%%%%%%%%%%%%%%%%%%%%%%%%%%%%%%%%%%%%%%%%%%%%%%%
\foilhead{Thanks}

\vspace*{-2ex}
{\small
\centerline{
\begin{tabular}{ccc}
% \includegraphics[width=1.3in]{colorado-reed} & 
% \includegraphics[width=1.4in]{miguel} &
% \includegraphics[width=1.3in]{neil-houlsby} &
\includegraphics[width=1.3in]{james-lloyd} &
\includegraphics[width=1.3in]{david-duvenaud} &
\includegraphics[width=1.5in]{roger-grosse} \\
% \includegraphics[width=1.5in]{koa-heaukulani} &
% \includegraphics[width=1.5in]{heller} &
% \includegraphics[width=1.5in]{shakir-mohamed} & 
% \PineGreen{Tom Griffiths} & 
% \PineGreen{Colorado Reed} & 
% \PineGreen{J.~Miguel Hernandez-Lobato} &
% \PineGreen{Neil Houlsby} &
\PineGreen{James Lloyd} &
\PineGreen{David Duvenaud} &
\PineGreen{Roger Grosse} \\
% \includegraphics[width=1.3in]{ryan-adams} & 
% \includegraphics[width=1.3in]{andrew-wilson} &
% \\
% \PineGreen{Ryan Adams} & 
% \PineGreen{Andrew Wilson} &
% \PineGreen{Creighton Heaukulani} &
% \PineGreen{Katherine Heller} & 
% \PineGreen{Shakir Mohamed} &
\end{tabular}
}
}

% \vspace*{2ex}

% \centerline{\tt http://learning.eng.cam.ac.uk/zoubin} 

% \vspace*{1ex}
% \centerline{\tt zoubin@eng.cam.ac.uk}

\vfill 

% Two ads:
% \begin{itemize}
{\small
\Blue{Duvenaud, D., Lloyd, J. R., Grosse, R., Tenenbaum, J. B. and
Ghahramani, Z. (2013) \PineGreen{Structure Discovery in Nonparametric
Regression through Compositional Kernel Search}. ICML 2013.} \\

\Blue{Lloyd, J. R., Duvenaud, D., Grosse, R., Tenenbaum, J. B. and
  Ghahramani, Z. (2013) 
\PineGreen{Automatic Construction and Natural-language Description of Additive
Nonparametric Models.}
NIPS workshop on Constructive Machine Learning} \\

\Blue{Ghahramani, Z. (2013) \PineGreen{Bayesian nonparametrics and
the probabilistic approach to modelling.} {\em
  Philosophical Trans. Royal Society A} 371: 20110553.}}

%%%%%%%%%%%%%%%%%%%%%%%%%%%%%%%%%%%%%%%%%%%%%%%%%%%%%%%%%%%%%%%%%%%%%%%%%
\foilhead{Appendix}

%%%%%%%%%%%%%%%%%%%%%%%%%%%%%%%%%%%%%%%%%%%%%%%%%%%%%%%%%%%%%%%%%%
\foilhead{Speeding up GP learning: Inducing point approximations}

\vspace*{-4ex}
\centerline{\PineGreen{\small (Snelson and Ghahramani, 2006)} }

\vspace*{1.5ex}
\Red{We can approximate GP through $M<N$ {\bf inducing points} $\bfb$} to
obtain the Sparse Pseudo-input Gaussian process (SPGP) a.k.a.\ FITC: 
%\begin{equation*}
$p(\bff) = \int\rd\bfb\,\prod_n p(f_n|\bfb)\; p(\bfb)$ \\[1ex]
%\end{equation*} 

\begin{minipage}[t]{0.2\linewidth}
\Green{GP prior}\\
\centering
$\cN(\bzero,\bKC{N})$
%\vspace{2ex}
%\includegraphics{priorcovnolabel}
\end{minipage}
%\hspace{0.06\linewidth}
\begin{minipage}[t]{0.06\linewidth}
\bigskip\medskip
$\approx$
\end{minipage}
\begin{minipage}[t]{0.7\linewidth}
\centering
\Green{FITC prior}\\
$p(\bff) \qquad\!\!=\; \cN(\bzero,\bKC{N\! M}\bKi{M}{-1}\bKC{M\! N} \;+ \quad\bLa)$
%\vspace{2ex}
%\includegraphics{SPGPpriorcov}
\end{minipage}
\includegraphics[width = 0.2\linewidth]{priorcovnolabel} 
\raisebox{65pt}{$\approx$}
\includegraphics[width = 0.2\linewidth]{SPGPpriorcov}
\raisebox{65pt}{=}
\includegraphics[height = 0.2\linewidth]{K}
\raisebox{133pt}{\includegraphics{invQ}}
\raisebox{134pt}{\includegraphics[width = 0.2\linewidth]{Kp}} 
\raisebox{65pt}{+}
\includegraphics[width = 0.2\linewidth]{factorizecovnolabel}

\begin{itemize}
\item FITC covariance inverted in $\cO(M^2N) \ll \cO(N^3) $
  $\Rightarrow$ \Blue{much faster}

\im FITC = GP with non-stationary covariance \Green{parameterized by
  $\bXb$}

\im Given data $\{\bX,\by\}$ with noise $\sigma^2$, predictive
  \Blue{mean} and \Red{variance} can be computed in \Blue{$\cO(M)$}
  and \Red{$\cO(M^2)$} per test case respectively

%% \begin{equation*}
%%   K_{\text{SPGP}}(\bx_n,\bx_{n'}) = \bK_{nM}\bK_M^{-1}\bK_{Mn'} 
%%   +  \lambda_n \delta_{nn'}
%% \end{equation*}
\end{itemize}

Builds on a large lit on sparse GPs \PineGreen{\small (see
  Qui\~{n}onero Candela and Rasmussen, 2006).}
%%%%%%%%%%%%%%%%%%%%%%%%%%%%%%%%%%%%%%%%%%%%%%%%%%%%%%%%%%%%%%%%%%
\foilhead{Speeding up GP learning: some developments since 2006}

\begin{itemize}
\item FITC (2006) 
\imm Unifying review \PineGreen{\small (see
  Qui\~{n}onero Candela and Rasmussen, 2006)}
\imm Combining local and global approximations \PineGreen{\small (w/
    Snelson, 2007)}
\imm Generalised FITC (for classification) \PineGreen{\small
    (Naish-Guzman and Holden, 2007)}
\imm Variational learning of inducing variables in sparse GPs \PineGreen{\small (Titsias, 2009)}
\imm Exploiting additive and Kronecker structure of kernels for very fast
  inference \PineGreen{\small (Saatci, 2011; Gilboa, Saatci,
    Cunningham 2013)} 
\imm GPatt: Fast Multidimensional Pattern Extrapolation with GPs \PineGreen{\small (Wilson, Gilboa, Nehorai, Cunningham,
  2013)} learns very flexible stationary kernels on 380k points using
Kronecker structure
\imm Gaussian Processes for Big Data \PineGreen{\small (Hensman,
    Fusi, Lawrence, 2013)} uses SVI to handle million data points
\end{itemize}

\end{document}




-----


%%%%%%%%%%%%%%%%%%%%%%%%%%%%%%%%%%%%%%%%%%%%%%%%%%%%%%%%%%%%%%%%%%
\foilhead{Some References}

\vspace*{-3ex}

{\small
\begin{itemize}

% \item Adams, R.P., Wallach, H., Ghahramani, Z. (2010) Learning the
% Structure of Deep Sparse Graphical Models.  AISTATS 2010.

\item Beal, M. J., Ghahramani, Z. and Rasmussen, C. E. (2002) The
 infinite hidden Markov model. NIPS {\bf 14}:577--585. 

 \item Bratieres, S., van Gael, J., Vlachos, A., and Ghahramani,
   Z. (2010) Scaling the iHMM: Parallelization versus
  Hadoop. International Workshop on Scalable Machine Learning and
  Applications (SMLA-10), 1235--1240.

\item Duvenaud, D., Lloyd, J. R., Grosse, R., Tenenbaum, J. B. and
Ghahramani, Z. (2013) Structure Discovery in Nonparametric
Regression through Compositional Kernel Search. ICML 2013.


\item Ghahramani, Z. (2013) Bayesian nonparametrics and
the probabilistic approach to modelling. {\em
  Phil. Trans. R. Soc. A} 371: 20110553.

% \item Griffiths, T.L., and Ghahramani, Z. (2006) Infinite Latent
% Feature Models and the Indian Buffet Process.
% NIPS  {\bf 18}:475--482. 

\item Griffiths, T.L., and Ghahramani, Z. (2011) The Indian buffet
  process: An introduction and review. {\em Journal of Machine
    Learning Research} {\bf 12}(Apr):1185--1224. 

\item Hernandez-Lobato, J.M, Houlsby, N. and Ghahramani, Z. (2013)
Stochastic Inference for Scalable Probabilistic Modeling of Binary
Matrices. NIPS 2013 Workshop on Randomized Methods for Machine
Learning. 
https://sites.google.com/site/randomizedmethods/rmml2013-schedule

\item Lloyd, J.R., Duvenaud, D., Grosse, R., Tenenbaum, J.B. and
Ghahramani, Z. (2013) Automatic Construction and Natural-language
Description of Additive Nonparametric Models. NIPS 2013 Workshop on
Constructive Machine Learning. 
http://www-kd.iai.uni-bonn.de/cml/program.html

%\item Knowles, D.A. and Ghahramani, Z. (2011) Nonparametric
%Bayesian Sparse Factor Models with application to Gene 
%Expression modelling. {\em Annals of Applied Statistics} {\bf
%  5}(2B):1534-1552. 
%
%\item Knowles, D.A. and Ghahramani, Z. (2011) Pitman-Yor Diffusion
%  Trees. In {\em Uncertainty in Artificial Intelligence (UAI
%   2011)}. % Oral

% \item Meeds, E., Ghahramani, Z., Neal, R. and  Roweis, S.T. (2007)
% Modeling Dyadic Data with Binary Latent Factors.  NIPS {\bf 19}:978--983. 

% \item Palla, K., Knowles, D.A., and Ghahramani, Z. (2012) An infinite
%  latent attribute model for network data. In ICML 2012.

\item Reed, C. and Ghahramani, Z. (2013) Scaling the Indian Buffet Process
via Submodular Maximization. ICML 2013.

\item Snelson, E., and Ghahramani, Z. (2007) Local and global sparse
  Gaussian process approximations. In {\em
    Eleventh International Conference on Artificial Intelligence and
    Statistics (AISTATS 2007)}. San Juan, Puerto Rico.

\item Snelson, E. and Ghahramani, Z. (2006) Sparse 
Gaussian Processes using Pseudo-Inputs.
In {\em Advances in Neural Information Processing Systems}
{\bf 18}:1257--1264. Cambridge, MA: MIT Press.


% \item Stepleton, T., Ghahramani, Z., Gordon, G., Lee, T.-S. (2009)
%  The Block Diagonal Infinite Hidden Markov Model. AISTATS 2009,
%  552--559.

%\item Wilson, A.G., and Ghahramani, Z. (2010, 2011) Generalised Wishart
%  Processes. arXiv:1101.0240v1. and UAI 2011

%\item Wilson, A.G., Knowles, D.A., and Ghahramani, Z. (2011) Gaussian
%  Process Regression Networks. arXiv.

\item Wilson, A.G, Gilboa, E., Nehorai, A., and Cunningham,
  J.P. (2013) GPatt: Fast Multidimensional Pattern Extrapolation with
  Gaussian Processes. arXiv:1310.5288

\item Wilson, A.G. and Adams, R.P. (2013)  Gaussian process covariance
  kernels for pattern discovery and extrapolation. International
  Conference on Machine Learning (ICML). 


%  \item van Gael, J., Saatci, Y., Teh, Y.-W., and Ghahramani, Z. (2008)
%  Beam sampling for the infinite Hidden Markov Model. ICML 2008, 1088-1095.

% \item van Gael, J and Ghahramani, Z. (2010) Nonparametric Hidden
%  Markov Models. In Barber, D., Cemgil, A.T. and Chiappa, S. {\em
%   Inference and Learning in Dynamic Models}. CUP. 

\end{itemize}
}

%%%%%%%%%%%%%%%%%%%


\newcommand{\uptext}[1]{\raisebox{2cm}{#1}}


\begin{tabular}{@{}c@{}}
\begin{tabular}{@{}cc@{}}


\includegraphics[trim=7.8cm 14.5cm 0cm 2cm, clip, width=0.5\columnwidth]{pg_0002-crop} 

 & \uptext{$=$} \\
\end{tabular}

\\
\begin{tabular}{p{7cm}p{0.1cm}p{7cm}p{0.1cm}p{7cm}}
\\%[1cm]
\includegraphics[trim=0.4cm 6cm 8.4cm 2.75cm, clip, width=0.3\columnwidth]{pg_0003-crop} & 
 \uptext{$+$}  &  
\includegraphics[trim=0.4cm 6cm 8.4cm 2.88cm, clip, width=0.3\columnwidth]{pg_0004-crop} & 
\uptext{$+$} &  
\includegraphics[trim=0.4cm 6cm 8.4cm 2.75cm, clip, width=0.3\columnwidth]{pg_0005-crop} \\
{\scriptsize A very smooth, monotonically increasing function }
& & 
{\scriptsize An approximately periodic function with a period of 1.0 years and with
approximately linearly increasing amplitude}
& & 
{\scriptsize An exactly periodic function with a period of 4.3 years but with linearly
increasing amplitude }
\end{tabular}
\end{tabular}

